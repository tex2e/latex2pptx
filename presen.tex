
\documentclass[dvipdfmx, 14pt]{beamer}
\usepackage{pgfpages}
\usepackage{bxdpx-beamer}

%%%% Viewerの文字化け対策 %%%%
\usepackage{atbegshi}
\ifnum 42146=\euc"A4A2
\AtBeginShipoutFirst{\special{pdf:tounicode EUC-UCS2}}
\else
\AtBeginShipoutFirst{\special{pdf:tounicode 90ms-RKSJ-UCS2}}
\fi
%%%%

%%%% スライドの見た目 %%%%
% Theme, Color, Font は一覧からお好みのものをどうぞ
% http://deic.uab.es/~iblanes/beamer_gallery/
\usetheme{Boadilla} % Theme
\usecolortheme{default} % Color
\usefonttheme{professionalfonts} % Font
\setbeamertemplate{headline}{
  \leavevmode%
  \vskip10pt% 上のマージン(狭すぎると見出しがスクリーンからはみ出る)
}
\setbeamerfont{frametitle}{size=\Large} % 見出しのサイズ
\setbeamertemplate{frametitle}[default][center] % 見出しの中央揃え
\setbeamertemplate{navigation symbols}{} % ナビゲーションの無効化
\setbeamerfont{footline}{size=\scriptsize, series=\bfseries} % footerのフォント
\setbeamertemplate{footline}{
  \leavevmode%
  \raggedleft% 右寄せ
  \insertframenumber{} / \inserttotalframenumber% ページ番号 / ページ総数
  \hspace*{5ex}% 右のマージン
  \vskip15pt% 下のマージン(狭すぎると行番号がスクリーンに映し出されない)
}
%%%%

%%%% フォントの基本設定 %%%%
\usepackage[utf8]{inputenc} % UTF-8
\usepackage[T1]{fontenc} % 8bit font
\usepackage{lmodern} % Latin Modern font
\usepackage{otf} % otf (font)
\usepackage{bm} % bold math
\usepackage{newtxtext, newtxmath} % newtx (font)
\renewcommand{\kanjifamilydefault}{\gtdefault} % fontfamily: Gothic
\usepackage{xcolor} % \textcolor{color}{text}
%%%%

%%%% tikz %%%%
\usepackage{tikz}
\usetikzlibrary{arrows}
\usetikzlibrary{shapes}
\usetikzlibrary{graphs}
\usetikzlibrary{backgrounds}
\usepackage{pgfplots} % plot graph
\pgfplotsset{compat=1.15}
%%%%

%%%% Table %%%%
\usepackage{booktabs}
\catcode63=\active \def?{\phantom{0}} % ? -> ' '
%%%%

%%%% Programming %%%%
\newcommand{\code}[1]{ \texttt{\detokenize{#1}} }
%%%%

\title{LaTeX + Beamer でプレゼン作成} % タイトル
\author{発表 太郎} % 名前
\institute{◯◯学科 ◯◯研究室} % 所属
\date{} % 日付

\begin{document}

% ==============================================================================
\begin{frame}[plain]
  \frametitle{}
  \titlepage %表紙
\end{frame}

% ==============================================================================
\begin{frame}
  \frametitle{表}

  \begin{itemize}
    \item 表は table と tabular を使います。
    \item キーとなる行と列は中央揃えにします。
    \item 小数点は phantom を使って縦に並ぶようにします。
  \end{itemize}

  \begin{table}
    \centering
    \begin{tabular}{cll} \toprule
        & \multicolumn{1}{c}{実測値} & \multicolumn{1}{c}{期待値} \\
      \midrule
      処理前 & 3.139710 & 3.141592 \\
      処理後 & 3.14???? & 3.14???? \\
      \bottomrule
    \end{tabular}
  \end{table}
\end{frame}

% ==============================================================================
\begin{frame}
  \frametitle{TikZの図(手順)}

  \begin{itemize}
    \item TikZの図は figure と tikzpicture を使います。
    \item 手順は tikz の graph を使うと良いです。
  \end{itemize}

  \begin{figure}
    \small
    \centering
    \begin{tikzpicture}[>=stealth, very thick]
      \tikzset{process/.style={
        rectangle, minimum size = 6mm, rounded corners = 3mm,
        very thick, draw = black!50,
        top color = white, bottom color = black!10,}
      }
      \graph [grow right sep] {
        "データ" -> "前処理"[process] -> "正規化"[process] ->[out=0, in=180] {
          "特徴選択"[process],
          "次元削減"[process]
        }
        ->[out=0, in=180] "分析"[process] -> "評価"[process]
      };
    \end{tikzpicture}
  \end{figure}
\end{frame}

% ==============================================================================
\begin{frame}
  \frametitle{TikZの図(階層構造)}

  \begin{itemize}
    \item TikZの図は figure と tikzpicture を使います。
    \item 階層構造は tikz の child を使うと良いです。
  \end{itemize}

  \begin{figure}
    \small
    \centering
    \begin{tikzpicture}[
        level 1/.style={sibling distance=11em},
        level 2/.style={sibling distance=11em},
        level 3 int/.style={sibling distance=3.5em},
        level 3 frac/.style={sibling distance=5em},]
      \tikzset{box/.style={
        rectangle, minimum size=6mm, rounded corners=3mm,
        very thick, draw=black!50,
        top color=white, bottom color=black!10,}
      }
      \node[box] {実数}
        child { node[box] {有理数}
          child { node[box] {整数}
            child[level 3 int] { node[box] {自然数} }
            child[level 3 int] { node[box] {0} }
            child[level 3 int] { node[box] {負の整数} }
          }
          child { node[box] {分数}
            child[level 3 frac] { node[box] {有限小数} }
            child[level 3 frac] { node[box] {循環小数} }
          }
        }
        child { node[box] {無理数} };
    \end{tikzpicture}
  \end{figure}
\end{frame}

% ==============================================================================
\begin{frame}
  \frametitle{TikZで数式の説明}

  \begin{itemize}
    \item 数式の一部分を説明するときには align の中で tikz を使います。
  \end{itemize}

  \tikzset{cover/.style={rectangle, rounded corners}}
  \begin{align*}
    \tikz[baseline=(x.base)]{
      \node (x) [label={ローレンツ力}]{
        $\vec{F}$
      };
    }
    =
    \tikz[baseline=(x.base)]{
      \node (x) [cover, fill=black!10, label={電気力}]{
        $q\vec{E}$
      };
    }
    +
    \tikz[baseline=(x.base)]{
      \node (x) [cover, fill=blue!10, label={磁気力}]{
        $q\vec{v} \times \vec{B}$
      };
    }
  \end{align*}

\end{frame}

% ==============================================================================
\begin{frame}
  \frametitle{TikZでOverlay}

  \begin{itemize}
    \setlength\itemsep{1cm}
    \item Foo Bar Baz Qux Quux Corge Grault Garply Waldo
    \item 要素をまたぐように矢印を描くには、tikz の overlay を使います。
    \item tikz の remember picture は使えません。
    \item Foo Bar Baz Qux Quux Corgi Grault Garply Waldo
  \end{itemize}

  \begin{tikzpicture}[overlay, >=stealth, very thick]
    \tikzset{ellipse_for_comp/.style={
      draw=blue!80, ellipse, text height=0.2cm, text width=0.7cm}}
    \coordinate (C) at (current page.south);
    \node[ellipse_for_comp] (comp old) at ([xshift=0.1cm, yshift=6.4cm]C) {};
    \node[ellipse_for_comp] (comp new) at ([xshift=0.1cm, yshift=1.0cm]C) {};
    \node (label) at ([xshift=2.5cm, yshift=4.0cm]C) {\small 不一致};
    \draw[<-, blue!80] (comp old) to (label.west);
    \draw[<-, blue!80] (comp new) to (label.west);
  \end{tikzpicture}
\end{frame}

% ==============================================================================
\begin{frame}
  \frametitle{TikZで折れ線グラフ}

  \vspace{1em}

  \begin{figure}
    \centering
    \begin{tikzpicture}
      \begin{axis}[
        height = 7cm,
        width  = 9cm,
        xlabel = $S_x$,
        ylabel = $S_y$,
        ]
        \addplot coordinates {
          (-4,38) (-3,22) (-2,3) (-1,-8) (0,1) (1,-4) (2,-7) (3,-16) (4,-42)
        };
        \addlegendentry{label1}
        \addplot coordinates {
          (-4,1) (-3,1.5) (-2,2.3) (-1,3.4) (0,5.1) (1,7.7) (2,11.5) (3,17) (4,21)
        };
        \addlegendentry{label2}
      \end{axis}
    \end{tikzpicture}
  \end{figure}
\end{frame}

% ==============================================================================
\begin{frame}
  \frametitle{TikZで棒グラフ}

  \vspace{1em}

  \begin{figure}
    \centering
    \begin{tikzpicture}
      \begin{axis}[
        xbar,
        enlargelimits = 0.15,
        symbolic y coords = {Foo,Bar,Baz,Qux,Corge},
        ytick = data,
        xlabel = $S_x$
        ]
        \addplot[fill=blue!80] coordinates {
          (10,Foo) (15,Bar) (5,Baz) (24,Qux) (30,Corge)
        };
      \end{axis}
    \end{tikzpicture}
  \end{figure}
\end{frame}

\end{document}
